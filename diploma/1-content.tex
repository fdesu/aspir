\section{Семейство ротационно-симметричных моделей звездных систем}

\subsection{Историческая справка}
Создание модели галактики --- трудоемкая и серьезная задача. Благодаря точно смоделированной системе представляются широкие возможности для исследований. К примеру, А. Оллонгрен в 1962 году занимался построением орбит исходя из существующей модели Галактики. Им было вычислено чрезвычайно много орбит, которые затем были классифицированы.~\cite{OLLONGREN}. Однако для того, чтоб приступить к построению орбит, сперва надо взять за основу модель, в которой эти орбиты мы будем строить. Задача моделирования галактики сводится к заданию ее потенциала. Здесь мы сводим задачу моделирования к отысканию функциии $\Phi(\textbf{r}, t)$ --- потенциала галактики, который будет в последствии хорошо отвечать реальным движениям небесных тел. Ниже приведены некоторые из наиболее известных потенциалов.\\
\newpage

\begin{center}
{
\begin{figure}[H]
\begin{table}\caption{Некоторые известные потенциалы}
  \begin{tabular}{| p{4cm}| c | p{4cm} | }  
    \hline
    Автор	&	Безразмерный потенциал $\Phi(r)$	&	Безразмерная плотность $\rho(r)$ \\ \hline
    Schuster(1883) Plummer(1911)		&	{\centering $(1 + r^2)^{-1/2}$ }	&	{\centering$3(1 + r^2)^{-5/2}$} \\ \hline
    H\'{e}non(1959) (isochrone model)	&	$\frac{2}{\displaystyle1 + \zeta(r)}$	&	$\frac{\displaystyle2\alpha^2(1 + 2\zeta)}{\displaystyle\zeta^3(1+\zeta)^2}$ \\
    \hline
  \end{tabular}
 \end{table}
 \end{figure}
}
\end{center}

Здесь $\zeta(r) = \sqrt{1 + \alpha^2r^2}$

\subsubsection{Потенциал Шустера-Пламмера}
Неоднократно делались попытки найти общие законы распределения плотности в звездных скоплениях. Как правило, при этом рассеянные и шаровые скопления рассматривались раздельно. Первые поиски общего закона распределения пространственной плотности звезд в звездных скоплениях были связаны с представлением о возможности проведения аналогии между звездными скоплениями и газовыми шарами. Пламмер (1911; 1916) аппроксимировал распределение плотности $\rho(r)$, наблюдаемое в ярких шаровых скоплениях, изученных Пиккерингом (1897) и Бэйли (1916), законом Шустера, предложенным последним для представления внутреннего строения звезд:

$$
	\rho(r) = \rho(0)\left[1 + \left(\frac{r}{r_0} \right)^2 \right]^{-5/2},
$$
где $r_0$ - некоторая постоянная. Популярности этого закона, по справедливому замечанию Костицына (1922), немало способствовала его простота. Поскольку закон Шустера - Пламмера удовлетворял наблюдениям далеко не всегда (Цейпель, 1913; Костицын, 1922), возникла необходимость его обобщения, впервые предложенного Валленквистом (1933):\\

\begin{equation}\label{vellenquist}
	\rho(r) = \rho(0)\left[1 + \left(\frac{r}{r_0} \right)^2 \right]^{-\beta},
\end{equation}
где $\beta$ - структурный параметр, меняющийся от скопления к скоплению. Согласно Ломану (1964), рассмотревшему распределение плотности в 67 рассеянных скоплениях по данным Валленквиста в 80\% случаев оно удовлетворительно представляется законом~(\ref{vellenquist}) при $2 < \beta < 4.4$\\

В 1959 г. Валленквист (1959) опубликовал результаты своей почти тридцатилетней работы по изучению структуры 67 рассеянных скоплений. Его исследование относится, в сущности, как мы теперь знаем, только к ядрам скоплений и даже в ряде случаев лишь к внутренним зонам этих ядер (за исключением Гиад). Для однородности исследования он подсчитывал звезды в пределах четырех величин, начиная от самых ярких.\\

Валленквист нашел, что распределение пространственной плотности в ядрах всех скоплений лучше всего представляется законом
$$
	\rho(r) = \rho(0)(1 - r)^c,
$$
где расстояние от центра $r$ выражается в долях радиуса скопления (вернее, в долях радиуса ядра), принятого Валленквистом: $0 \leq 1 - r \leq 1$. Постоянная с называется степенью концентрации. При $c = 0\quad \rho(r)$ не зависит от $r$ и равна центральной плотности $\rho(0)$. Чем больше $с$, тем быстрее изменяется $\rho(r)$, тем сильнее концентрируются звезды в ядре скопления к его центру~\cite{SHUSTER_PLUMMER}.


\subsubsection{Потенциал Энона}
Энон нашел наиболее общий случай потенциала, где радиальный период орбит зависит лишь от энергии. Автор назвал свой потенциал изохронным и нашел, что он дает хорошее представление моделей шаровых скоплений. Энон так же дал объяснение в терминах резонансной релаксации. Роль, которую резонансная релаксация может сыграть в шаровых скоплениях, до сих пор загадка. Однако, изохронный потенциал гарантирует себе место в звездной динамике, потому что это наиболее общая форма потенциала в которой доступны выражения в неявном виде для угловых моментов (angle-actions). В статье~\cite{HENON}  Бинни объясняет, как это свойство делает модель Энона бесценной для мощных методов тороидального отображения. Так же он исследует сплюснутые изохронные модели, которые позволяют открыть мощный общий метод генерации непротиворечивых звездных систем.

%{
%\begin{figure}[H]
%\centering
%\begin{minipage}[t]{0.8\textwidth}
%\centering
%\includegraphics[width=\linewidth]{D:/diploma/tex/1/res/fig1.jpg}
%\end{minipage}
%\end{figure}
%}
%{
%\begin{figure}[H]
%\centering
%\begin{minipage}[t]{0.8\textwidth}
%\centering
%\includegraphics[width=\linewidth]{D:/diploma/tex/1/res/fig2.jpg}
%\end{minipage}
%\end{figure}
%}
%{
%\begin{figure}[H]
%\centering
%\begin{minipage}[t]{0.8\textwidth}
%\centering
%\includegraphics[width=\linewidth]{D:/diploma/tex/1/res/fig3.jpg}
%\end{minipage}
%\end{figure}
%}
%{
%\begin{figure}[H]
%\centering
%\begin{minipage}[t]{0.8\textwidth}
%\centering
%\includegraphics[width=\linewidth]{D:/diploma/tex/1/res/fig4.jpg}
%\end{minipage}
%\end{figure}
%}

\subsection{Описание параметрического семейства моделей}

~\par
Гравитационное поле потенциально. Это значит, что можно ввести потенциальную энергию гравитационного притяжения пары тел, и эта энергия не изменится после перемещения тел по замкнутому контуру. Потенциальность гравитационного поля влечёт за собой закон сохранения энергии и при изучении движения тел в гравитационном поле часто существенно упрощает решение. В рамках ньютоновской механики гравитационное взаимодействие является дальнодействующим. Таким образом, как бы массивное тело ни двигалось, в любой точке пространства гравитационный потенциал зависит только от положения тела в данный момент времени~\cite{LandauLifshitz}. Гравитационное поле в моделях галактик обычно описывается с помощью потенциала как функции координат (а также времени, если система нестационарна). Закон изменения потенциала $\varphi (r)$ взят из работы~\cite{RaspopvaOssipkovJiang}. В качестве потенциала $\varphi$ понимается скалярная функция координат, характеризующая гравитационное поле.  Введем сферическую систему координат $(r, \theta, \chi)$ и рассмотрим закон изменения $\varphi$, когда он зависит лишь от $r$, где $r$ --- расстояние от начала координат до пробной звезды.\\

\begin{equation}
\varphi (r) = \frac{\alpha}{\alpha - 1 + w(r)},\,  w(r) = (1+\kappa r^p)^{1/p},\, \kappa = O(\alpha ^p),
\end{equation}
где
\begin{center}
    \begin{tabular}{l c p{0.9\linewidth}}
$\alpha,\, \kappa,\, p$   & --- & структурные параметры, \\
$\alpha > 0,\, p > 0$     & --- & ограничения, накладываемые на структурные\\параметры~\cite{RaspopvaOssipkovJiang}.\\
  \end{tabular}
\end{center}

Покажем, что потенциал не имеет каких-либо особенностей в нуле и на бесконечности. Для этого рассмотрим два предела:

\begin{equation}
	\lim_{r \to 0} \varphi(r) = \lim_{r \to 0} \frac{\alpha}{\alpha - 1 + (1 + O(\alpha^p)r^p)^{\frac{1}{p}}} = 1,
\end{equation}
\begin{equation}
	\lim_{r \to \infty}\varphi(r) = \lim_{r \to \infty} \frac{\alpha}{\alpha - 1 + (1 + O(\alpha^p)r^p)^{\frac{1}{p}}} = 0.
\end{equation}
отсюда можно заключить, что потенциал в нуле и на бесконечности без особенностей.

~\par
Ротационная симметрия  означает симметрию объекта относительно всех или некоторых собственных вращений $m$-мерного евклидова пространства. Выберем 3-мерное евклидово пространство и цилиндрическую систему координат $(R, \theta, z)$, с плоскостью симметрии $z=0$. Так как ранее использовалась сферическая система координат, сделаем переход к цилиндрической системе:

\begin{equation}
\left\{ \begin{aligned}
& R = r \sin \chi, \\
& \theta = \theta, \\
& z = r \cos \chi.
\end{aligned}
\right.
\end{equation}

Когда рассматривалась цилиндрическая система координат, в уравнении потенциала изменялась только сферическая координата $r$, а следовательно $\theta = \text{const}, \chi = \text{const}$. При переходе к цилиндрическим координатам, изменяется $$w(r) = (1+\kappa (r)^p)^{1/p} = (1+\kappa (R/\sin \chi)^p)^{1/p},$$ где $\sin \chi = \text{const} $ и учтена в параметре $\kappa$.\\

Рассмотрим обобщение на ротационно-симметричную модель с помощью эквипотенциали (эквипотенциаль --- это линия, на которой скалярный потенциал данного потенциального поля принимает постоянное значение) Миямото и Нагаи~\cite{MiamotoNagai}. Выберем закон потенциала $\varphi:$\\

\begin{equation}\label{eq:xi}
        \varphi (\xi ) =  \frac{\alpha}{\alpha - 1 + w(\xi )} \text{,}\, w(\xi ) = (1 + \kappa \xi ^p)^{1/p},\, \kappa = O(\alpha ^p),
\end{equation}
\begin{equation}
        \xi ^2 = R^2 + z^2 + 2(1 - \varepsilon)(\sqrt{z^2 + \varepsilon ^2} - \varepsilon).
\end{equation}
Здесь $\varepsilon \in [0,1] $ --- структурный параметр.

\subsection{Единицы измерения}

Все использующиеся переменные являются безразмерными. Для того, чтобы перейти от безразмерных единиц к размерным, необходимо домножить на соответствующий размерный коэффициент. Эталонными размерными единицами являются единицы длины, массы и потенциала:

$$
\widehat{r}, \quad 
\widehat{m} = \frac{\widehat{r}\widehat{\varphi}}{G}, \quad 
\widehat{\varphi}.
$$

Производными единицами скорости, потенциала и пространственной плотности масс являются соответственно:

$$
\widehat{v} = \widehat{r}\widehat{t}^{-1}, \quad
\widehat{\rho} = \widehat{r}^{-3}\widehat{m}.
$$
В качестве размерной единицы времени возьмем период $T_c$ движения по круговой орбите с наибольшей скоростью $v_c$. Эта единица следует из основных:

$$
T_c = \frac{2\pi r_c}{v_c}, \quad \widehat{t} = \frac{\widehat{r}}{\sqrt{\widehat{\varphi}}}.
$$

Эти соотношения не изменяют вида формул, в которые входят соответсвующие размерные величины. Единица гравитационной постоянной следует из известного выражения для потенциала материальной точки:

$$
\widehat{G} = \frac{\widehat{r}\widehat{\varphi}}{\widehat{m}} = \widehat{r}^3\widehat{m}^{-1}\widehat{t}^{-2}.
$$

И если в качестве $\widehat{G}$ взять значение самой гравитационной постоянной, то безразмерная постоянная будет равна единице, а на основные единицы будет наложена связь. Сама постоянная исчезнет из всех уравнений.\\



\subsection{Потенциал при различных значениях параметров}

~\par
Начнем исследования с рассмотрения начального набора параметров: $\alpha = 2, p = 1.5, \varepsilon = 0.5$, что позволит получить из трехпараметрического семейства потенциалов частную модель для исследований. Схема рассмотрения такова, что сначала будем фиксировать два параметра, а оставшийся изменять в допустимых пределах. Это позволит лучше понять каким образом отдельные параметры влияют на модель.\par
На рис.~\ref{equipot} представлены меридиональные сечения эквипотенциалей при различных значениях параметра $\varepsilon$. Меридиональные сечения --- это те сечения, которые получены сечением плоскостью проходящей через ось вращения тела.

{
\begin{figure}[H]
    \centering
    \includegraphics[scale = 0.65]{D:/diploma/tex/1/res/equipot.png}
    \caption{Меридиональные сечения эквипотенциали при различных $\varepsilon$}\label{equipot}
\end{figure}
}
{\flushleft На графике видно, что при уменьшении значения $\varepsilon$ в заданных пределах, меридиональное сечение эквипотенциали принимает форму овальной кривой, причем при $\varepsilon=1$ получаем кривую, которая сходна с частью окружности (часть окружности с центром в точке (0,0) в первой четверти), а при уменьшении значения $\varepsilon$ получившаяся кривая более сплюснута по вертикали~\cite{RaspopvaOssipkovJiang}.\par}
Далее рассмотрим вопрос о том, как структурные параметры $\alpha, p$ влияют на потенциал, заданный формулой~(\ref{eq:xi}). Проварьируем их и получим результаты, показанные на рисунках \ref{pot:z}, \ref{pot:R}:

\begin{figure}[H]
\centering
\begin{minipage}[t]{0.49\textwidth}
\centering
\includegraphics[width=\linewidth]{D:/diploma/tex/1/res/alpha(z).png}
\end{minipage}\hfill
\begin{minipage}[b]{0.49\linewidth}
\centering \includegraphics[width=\linewidth]{D:/diploma/tex/1/res/p(z).png}
\end{minipage}
\caption{Ход потенциала на оси $R=0$ в безразмерных единицах}\label{pot:z}
\end{figure}

{\flushleft На рис.~\ref{pot:z} представлен ход потенциала на оси $R=0$. Видно, что при уменьшении $\alpha$ график потенциала становится более пологим. При уменьшении $p$ график потенциала становится более крутым.
}

На представленных графиках изображен ход потенциала на оси $R=0$. Видно, что при уменьшении $\alpha$ график потенциала становится более пологим. При уменьшении $p$ график потенциала становится более крутым.
\begin{figure}[H]
\centering
\begin{minipage}[t]{0.49\textwidth}
\centering
\includegraphics[width=\linewidth]{D:/diploma/tex/1/res/alpha(R).png}
\end{minipage}
\hfill
\begin{minipage}[b]{0.49\linewidth}
\centering \includegraphics[width=\linewidth]{D:/diploma/tex/1/res/p(R).png}
\end{minipage}
\caption{Ход потенциала в плоскости $z=0$ в безразмерных единицах}\label{pot:R}
\end{figure}
{\flushleft На рис.~\ref{pot:R}  видно, что при уменьшении $\alpha$ график потенциала становится более пологим. При уменьшении $p$ график потенциала становится более крутым.
}

\subsection{Влияние параметров на ход плотности}
~\par
Плотность --- скалярная физическая величина, определяемая как отношение массы тела к занимаемому этим телом объёму или площади (поверхностная плотность). Рассмотрим плотность вещества --- плотность тела, состоящего из этого вещества, где тело --- галактика, порождающая потенциал $\varphi$. В общем случае, когда плотность вещества $\rho$ распределена произвольно, $\varphi$ определяется как решение уравнения Пуассона (в безразмерных величинах):\\
\begin{equation}
\triangle \varphi = -4\pi G\rho, \text{ где  } \triangle \text{ --- лапласиан}.\\
\end{equation}
~\par
 Учитывая ротационную симметрию рассматриваемой модели и выражение для лапласиана в цилиндрических координатах, имеем выражение для плотности в безразмерных величинах:
 \begin{equation}
 \rho (R,z) = -\frac{\partial ^2\varphi}{\partial R^2}
 -\frac{\partial \varphi}{R\partial R}
 -\frac{\partial ^2\varphi}{\partial z^2} = -\triangle \varphi
 \end{equation}
 
 Вычисляя выражение для плотности $\rho (R,z)$ мы приходим к таким уравнениям:
 
 \begin{equation}
 \begin{split}
\rho (R,z) = \alpha\left( \frac{\displaystyle\frac{\displaystyle\partial^2w(R,z) }{\displaystyle\partial R^2}}{(\displaystyle\alpha - 1 + w(R,z))^2} + \frac{\displaystyle\frac{ \partial^2w(R,z) }{\partial z^2}}{\displaystyle(\alpha - 1 + w(R,z))^2}\right) +\\ 
\frac{\alpha}{R} \left(\frac{\displaystyle\frac{ \partial w(R,z) }{\partial R}}{\displaystyle(\alpha - 1 + w(R,z))^2}\right) - \\ 
2\alpha \left( \frac{\displaystyle \left(\frac{ \partial w(R,z) }{\partial R}\right)^2}{\displaystyle(\alpha - 1 + w(R,z))^3} + \frac{\displaystyle\left(\frac{ \partial w(R,z) }{\partial z}\right)^2}{\displaystyle(\alpha - 1 + w(R,z))^3}\right),\text{~}
 \end{split}
 \end{equation}
 
 \begin{equation}
 \begin{split}  
&\frac{\partial w(R,z)}{\partial R} = \alpha ^p \xi (R,z)^{p-1} \frac{\partial \xi}{\partial R} \left( \alpha^p \xi(R,z)^p +1  \right)^{1/p -1}, \\
&\frac{\partial w(R,z)}{\partial R} = \alpha ^p \xi (R,z)^{p-1} \frac{\partial \xi}{\partial z} \left( \alpha^p \xi(R,z)^p +1  \right)^{1/p -1},
 \end{split}
 \end{equation}
 
 \begin{equation}
 \begin{split}  
\frac{ \partial^2w(R,z) }{\partial R^2} = \alpha^p \xi(R,z)^{p-1} \frac{ \partial^2\xi(R,z) }{\partial R^2} \cdot (\alpha^p \xi(R,z)^p + 1) ^ {1/p - 1} + \\
+\, \alpha^p\xi^{p-2} \left( \frac{\partial \xi}{\partial R} \right)^2 (p-1) (\alpha^p \xi^p + 1)^{1/p - 1}, \\
\frac{ \partial^2w(R,z) }{\partial z^2} = \alpha^p \xi(R,z)^{p-1} \frac{ \partial^2\xi(R,z) }{\partial z^2} \cdot (\alpha^p \xi(R,z)^p + 1) ^ {1/p - 1} + \\
+\, \alpha^p\xi^{p-2} \left( \frac{\partial \xi}{\partial z} \right)^2 (p-1) (\alpha^p \xi^p + 1)^{1/p - 1},
 \end{split}
 \end{equation}
 
  \begin{equation}
 \begin{split}  
&\frac{\partial \xi(R,z)}{\partial R} =  2R,\\
&\frac{\partial \xi(R,z)}{\partial z} = 2z \left( 1 - \frac{\varepsilon - 1}{\sqrt{\varepsilon^2 + z^2}} \right),
 \end{split}
 \end{equation}
 
  \begin{equation}
 \begin{split}  
&\frac{ \partial^2\xi(R,z) }{\partial R^2} = 2,\\
&\frac{ \partial^2\xi(R,z) }{\partial z^2} = 2\left( 
z^2 \frac{\varepsilon - 1}{(\varepsilon^2 + z^2)^{3/2}} - \frac{\varepsilon - 1}{\sqrt{\varepsilon^2 + z^2}}  +1 
\right),
 \end{split}
 \end{equation}

Так же как и в предыдущем пункте рассмотрим вариации структурных параметров и соответствующие им изменения в модели распределения массы в рассматриваемой нами модели.
\begin{figure}[H]
\centering
\begin{minipage}[t]{0.49\textwidth}
\centering
\includegraphics[width=\linewidth]{D:/diploma/tex/1/res/rho(z,a).png}
\end{minipage}
\hfill
\begin{minipage}[b]{0.49\linewidth}
\centering \includegraphics[width=\linewidth]{D:/diploma/tex/1/res/rho(z,p).png}
\end{minipage}
\caption{Ход плотности на оси $R=0$ в безразмерных единицах}\label{rho:z}
\end{figure}
{\flushleft Из рис.~\ref{rho:z} (слева), рис.~\ref{rho:R} (слева) видно, что с увеличением значения $\alpha$ пик плотности становится выше, то есть значение плотности в центре координат принимает большие значения (плотность остается конечной, а не уходит на бесконечность, как может показаться из рисунков~\ref{rho:z} и ~\ref{rho:R}). На рис.~\ref{rho:z} (справа), рис.~\ref{rho:R} (справа) вариации параметра $p$ привели к более интересным изменениям хода плотности. Как мы видим при $p>2$ значение плотности в центре принимает нулевое значение, при $p<2$ имеем, опять же, центральный пик плотности, при $p=2$ плотность конечна.}
{
\begin{figure}[H]
\centering
\begin{minipage}[t]{0.49\textwidth}
\centering
\includegraphics[width=\linewidth]{D:/diploma/tex/1/res/rho(R,a).png}
\end{minipage}
\hfill
\begin{minipage}[b]{0.49\linewidth}
\centering \includegraphics[width=\linewidth]{D:/diploma/tex/1/res/rho(R,p).png}
\end{minipage}
\caption{Ход плотности в плоскости $z=0$ в безразмерных единицах}\label{rho:R}
\end{figure}
}


\begin{figure}[H]
\centering
\begin{minipage}[t]{0.49\textwidth}
\centering
\includegraphics[width=\linewidth]{D:/diploma/tex/1/res/dens.png}
\end{minipage}
\caption{При некоторых  значениях $\varepsilon$ получается отрицательная плотность}\label{rho:unreal}
\end{figure}
{\flushleft Как видно из рис.~\ref{rho:unreal}, не все значения структурных параметров нам подходят. Так как конечной целью нашей работы является построение орбит, то и модель, которая порождает эти орбиты, должна быть физически корректна, то есть плотность должна быть всюду неотрицательна.}


~\\
~\\
{
\begin{figure}[H]
\centering
\begin{minipage}[t]{0.45\textwidth}
\centering
\includegraphics[width=\linewidth]{D:/diploma/tex/1/res/equidens.png}
\end{minipage}
\caption{Меридиональные сечения эквиденсит для выбранных структурных параметров}\label{rho:eqidens}
\end{figure}
}
{\flushleft Рис.~\ref{rho:eqidens} показывает нам характер распределения масс в рассматриваемой системе, где линиям разного начертания соответсвуют различные значения плотности. Здесь же можно видеть, что вещество, заключенное в нашей системе, сосредоточено преимущественно в центре.}\par
В первой главе была приведена историческая справка, которая объяснила, что моделирование галактик позволяет извлечь много ценной информации не только о самой галактике, но и о процессах, происходящих в ней. Было описано трехпараметрическое семейство потенциалов, которое исследуется в работе. Приведены единицы измерения, которые позволяют рассматривать процессы, происходящие в модели, в размерных единицах, что может быть полезно, при применении модели к какой-либо реально существующей галактике. Показано, что параметры вляют на рельеф сил так, что его форма в отдельных местах становится более крутой или более пологой. Также в ходе исследования плотности показано, что не все параметры нам подходят: некоторые параметры не только увеличивают или уменьшают пик плотности в центре, но и даже делают плотность в центре отрицательной. Полученные результаты позволяют заключить, что набор параметров $\alpha = 2, p = 1.5, \varepsilon = 0.5$ определяет физически корректную модель, с которой мы уже можем работать и в которой мы уже можем строить орбиты.
