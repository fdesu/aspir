\section*{Введение}
\addcontentsline{toc}{section}{Введение}
~\par
Каждая звезда под действием лишь регулярных сил должна описывать внутри стационарной звездной системы неизменную (т.е. не зависящую от времени) регулярную орбиту. С этой точки зрения динамика звездной системы определяется распределением регулярных орбит, что, впрочем, эквивалентно заданию координат и скоростей всех звезд данной системы (заданию фазового распределения) для какого-нибудь момента времени $t_0$, который может быть принят за <<начальный>> момент. Для многих изучение динамики началось и закончилось вторым законом Ньютона $F=mA$. Считается, что если заданы силы, действующие между звездами, а так же начальные положения и скорости звезд, то с помощью компьютера можно предсказать движение или развитие системы для любого сколь угодно позднего момента времени. Однако, таковые расчеты на компьютерах не привели к обещанной предсказуемости в динамике. Напротив, было обнаружено, что движение некоторых очень простых динамических систем не всегда можно предсказать на большой интервал времени. Такие движения были названы \textit{хаотическими}, и их исследования привлекли в динамику некоторые новые математические идеи. Термин <<хаотический>> применяется в тех детерминированных задачах, где отсутствуют случайные или непредсказуемые силы и параметры. Таковой является исследуемая система и на ее примере рассматривается полный цикл от проверки параметров на физическую корректность, до исследования порожденных орбит с помощью различных методов на хаотичность.\\

{\large\textbf{Цель: исследование рассматриваемой модели, построение орбит и их анализ на хаотичность.}}\\

Дипломная работа состоит из введения, трех глав, заключения и списка литературы, содержащего 12 наименований. В первой главе говорится об исследуемой модели. Приводится историческая справка, модели, полученные другими авторами, их классификация и краткое пояснение. Так же проводится анализ исследуемой модели: первой и второй производной, проверяется физическая корректность структурных параметров и утверждение их в качестве используемых в работе. Вторая глава посвящена приведению методов численного интегрирования, рассмотрению системы дифференциальных уравнений, из которых получаются орбиты, преобразовани ее с помощью интегралов движения, а так же само построение орбит. Третья глава посвящена анализу полученных орбит на хаотичность различными методами: сечениями Пуанкаре, фрактальной размерностью, а так же показателями Ляпунова.