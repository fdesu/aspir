\section*{Заключение}
\addcontentsline{toc}{section}{Заключение}
~\par
В представленной работе была рассмотренная обобщенная ротационно-сим\-мет\-рич\-ная модель звездной системы и разработан метод анализа хаотичности.\\

Рассмотрено трехапараметрическое семейство потенциалов и исследованная ассимптотика в нуле и на бесконечности, и сделан вывод, что выбранный потенциал не имеет особенностей. Введены единицы измерения для того, чтобы было возможным рассмотренную модель применить к реально существующим галактикам. Исследованно влияние структурных параметров модели на рельеф сил, которое проявилось в изменении его формы в отдельных местах. Получено выражение для распределения пространственной плотности, в ходе исследования которого выявлены дополнительные ограничения на структурные параметры, исходя из условия физической корректности. Построены эквиденситы, говорящие о том, что вещество, заключенное в системе, сосредоточено преимущественно в центре.\\

Рассмотрена задача Коши, в результате решения которой был получен набор орбит (около 1000). Был выбран метод численного интегрирования Рунге-Кутта-Фельберга и соответствующие значения относительной и абсолютной погрешностей для того, чтобы получить достаточно точные решения. Проведен выбор начальных условий с помощью построенной диаграммы Линдблада, которая показала, что выбранные начальные условия для изображенных в работе орбит являются допустимыми. Построен и проанализирован набор орбит, а также сделан вывод, что большинство вычисленных орбит оказались ящичного типа.\\

Для вычисленных орбит построены соответствующие сечения Пуанкаре, а также сделан вывод, что сами по себе сечения Пуанкаре в определенных случаях не являются достаточным критерием для выявления нерегулярности выбранной орбиты. Для более точной картины вычислены соответствующие фрактальные размерности, которые хоть и оказались дробными, но близкими к целому значению. Разработан алгоритм вычисления фрактальных размерностей и выполнены предварительные расчеты. Построенный набор орбит позволил заключить, что малым изменениям начальных условий при решении задачи Коши отвечает малое изменение построенных орбит.

\begin{thebibliography}{4.}

\bibitem{KutuzovOssipkov} Кутузов С.А., Распопова Н.В. Рельеф поля сил и орбиты в модели галактики // Вестник СПбГУ. Сер. 10. 2008. вып. 1. С. 32-42.

\bibitem{LandauLifshitz} Ландау Л.Д., Лифшиц Е.М. Теоретическая физика: Учеб. пособ.: Для вузов. В 10 т. Т.II. Теория поля.. 8 изд. М: ФИЗМАТЛИТ, 2003.  536~с.

\bibitem{Mun} Мун~Ф. Хаотические колебания: Вводный курс для научных работников и инженеров: Пер. с англ. М.: Мир, 1990. 312\,с.

\bibitem{ogor} Огородников~К.\:Ф. Динамика звездных систем. М: Физматгиз, 1958. 644~с.

\bibitem{SHUSTER_PLUMMER} Общие законы распределения плотности в звездных скоплениях. Закон Шустера. Закон Кинга // Астронет URL: \url{http://www.astronet.ru/db/msg/1246874/8.9.html}

\bibitem{RaspopvaOssipkovJiang}  Raspopova~N.V., Ossipkov~L.P., Jiang~Z. A New Model for Dark Halo of Spherical Galaxies // Astronomical and Astrophysical Transactions. 2012. Vol 27. Issue 3, pp 433-436.

\bibitem{MiamotoNagai} Miyamoto,~M., Nagai,~R. Three-dimensional models for the distribution of mass in galaxies // Astronomical Society of Japan, Publications. 1975. vol. 27, no. 4. p.~533-543.

\bibitem{OLLONGREN} A. Ollongren, BAN, 16, 241 (1962).

\bibitem{BinneyTremaine} Binney~J., Tremaine~S Galactic dynamics. Princeton:  Princeton University Press, 2008. 904~p.

\bibitem{HENON} Henon's Isochrone Model // arXiv.org URL: \url{http://arxiv.org/abs/1411.4937}

\bibitem{KAPLAN_AND_YORKE} Kaplan~J.L., and Yorke~J.A. Chaotic behavior of multidimensional difference equations // Functional Differential Equations and the Approximation of Fixed Points, Lecture Notes in Mathematics. - vol. 730, H.O. Peitgen and H.O. Walther, eds. (Springer, Berlin). p~ 228.

\bibitem{MINKOWSKI} Minkowski–Bouligand dimension // Wikipedia URL: \url{http://en.wikipedia.org/wiki/Minkowski-Bouligand_dimension}

\end{thebibliography}